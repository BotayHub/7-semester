\documentclass[a4paper,12pt]{report}
\usepackage[T2A]{fontenc}
\usepackage[utf8]{inputenc}
\usepackage[english,russian]{babel}
\usepackage{graphicx}
\usepackage{wrapfig}
\usepackage{mathtext} 				% русские буквы в фомулах
\usepackage{amsmath,amsfonts,amssymb,amsthm,mathtools} % AMS
\usepackage{icomma} % "Умная" запятая: $0,2$ --- число, $0, 2$ --- перечисление
\usepackage{capt-of}
\usepackage{appendix}
\usepackage{multirow}
\usepackage{hyperref}
\usepackage{floatrow}
\usepackage[left=2cm,right=2cm,
    top=2cm,bottom=2cm,bindingoffset=0cm]{geometry}
\usepackage{multicol} % Несколько колонок
\usepackage{gensymb}
\title{Отчёт по лабораторной работе №1

Изучение особенностей возбуждения и распространения акустических волн СВЧ в твердых телах}
\author{Плюскова Н., \\
Шарапов А., \\
Пасько И.}
\date{\today}

\begin{document}

\maketitle


\section*{1. Теоретические данные}
Под затуханием ультразвуковых волн (УЗВ) обычно понимают уменьшение интенсивности вдоль пути ее распространения. Это связано со следующими процессами: поглощением энергии УЗВ и переходом ее в тепло, с рассеянием на неоднородностях и причинами, создающими кажущееся поглощение, связанное с методикой измерений, к примеру, разориентации образца относительно основных кристаллографических осей, дифракционные потери, потери из-за не параллельности торцевых граней образца и другие.

Первые две причины создают уменьшение интенсивности, пропорциональные самой интенсивности, то есть $-dI(x)=\gamma I(x)dx$ или $I(x)=I_{0}e^{-\gamma x}$. Для амплитуд выражение имеет вид $U(x)=U_{0}e^{-\alpha x}$. $U_{0}, I_{0}$ – интенсивность и амплитуда УЗВ во входном сечении кристалла. $\alpha$ – коэффициент затухания амплитуды, а $\gamma=2\alpha$ – коэффициент затухания интенсивности. Если при измерении затухания амплитудные характеристики линейны, то для определения $\alpha$ можно использовать следующее выражение:
\begin{equation*}
\alpha=-\frac{1}{x_{1}-x_{2}}ln\frac{U(x_{1})}{U(x_{2})}
\end{equation*}
Если регистрация амплитуды УЗВ происходит в одном и том же сечении образца, то $x_{2}-x_{1}=2L$, где $L$ – длина образца, а величину можно найти, измеряя отношение амплитуд соответствующих импульсов на экране осциллографа. На этом и основан реализуемый в работе метод.

В работе на одном из двух торцов образца мы возбуждаем УЗВ, распространяющиеся вглубь образца. Переменное электрическое поле прикладывается к преобразователю на очень короткое время (порядка нескольких микросекунд). В результате по кристаллу распространяется короткий цуг УЗВ длиной $V_{s}\tau_{\text{имп}}$, где $V_{s}$ – скорость УЗВ. Испытав отражение от параллельной грани и придя обратно, цуг вызывает но обкладках преобразователя переменное напряжение с частотой УЗВ. На выходе мы наблюдаем импульс длиной $\tau_{\text{имп}}$. Скорость УЗВ мы находим временную задержку n-го импульса относительно m-го. Эта задержка соответствует целому числу двойных пробегов цуга УЗВ вдоль образца, поэтому $V_{s}=\frac{2L(m-n)}{T_{3}}$.
	

    \section*{2. Экспериментальная часть}
\begin{figure}[H]
	\centering
	\includegraphics[scale=0.6]{scheme.pdf}
	\caption{Схема установки}
\end{figure}
    
\section*{3. Результаты эксперимента и обработка данных}
На частоте 420 МГц измерим коэффициент затухания амплитуды УЗВ и скорость УЗВ в кристаллах $SiO_{2}$ и двух образцах $YAG$:
\begin{table}[H]
\begin{tabular}{|c|c|c|}
\hline
Образец  & $\alpha, \frac{\text{дБ}}{\text{см}}$ & $V_s, \frac{\text{см}}{\text{мкс}}$ \\ \hline
SiO2     & 0,39         & 0,73         \\ \hline
Гранат 1 & 0,42         & 1,20         \\ \hline
Гранат 2 & 1,59         & 0,40         \\ \hline
\end{tabular}
\end{table}

Исходя из полученных данных, определим константы упругости второго порядка:
\begin{center}
    $C_{11} = \rho \cdot {V_s}^2$ = 0,12 $\frac{\text{кг}}{\text{м}\cdot c^2}$ - Кремний

    КАК ДЛЯ ГРАНАТА НАЙТИ КОМБИНАЦИИ КОНСТАНТ УПРУГОСТИ?
\end{center}


Снимем частотную зависимость $\alpha(\nu)$ в $SiO_2$ и построим соответствующий график в двойном логарифмическом масштабе:

ГРАФИК

Проведем расчет $\Delta_{\text{диф}}$ на $\nu=400$ МГц по формуле:
\begin{equation}
    \Delta_{\text{диф}}=20 \log(\frac{\lambda l}{\pi a^2})\cdot\frac{\sin(\frac{\lambda l}{\pi a^2}\cdot\frac{\pi}{3.83})^4}{(\frac{\lambda l}{\pi a^2}\cdot\frac{\pi}{3.83})^4} = 
\end{equation}

Радиус преобразователя приближенно равен $a=0.05$ см, $l=2 L$, $\lambda_{\text{з}}=\frac{l/t}{400\text{МГц}}=625$ нм


\section*{4. Выводы}
В ходе работы была снята частотная характеристика коэффициента затухания амплитуды УЗВ, на частоте 420 МГц определены скорости УЗВ в кремнии и двух образцах граната, определены комбинации констант упругости 2-го порядка всех образцов. Также были оценены дифракционные потери в кристалле кремния.

\end{document}
